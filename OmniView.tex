\documentclass{scrreprt}
\usepackage{glossaries}
\usepackage{listings}
\usepackage[]{graphicx}

\newglossaryentry{latex}{
  name=LaTeX,
  description={Ein Textsatzsystem}
}
\newglossaryentry{consumer}{
    name=Consumer,
    description={A program using a UaDI conform DLL to get generated data from a producer or device}
}
\makeglossaries

\tableofcontents
\newpage 

\begin{document}

\chapter{The Idea of OmniView}
OmniView is planned to be an omniscient data-visualization and analysis tool. 
OmniView itself shouldn't need to know anything about any given data-producer at compile-time, but should still be able to visualize the produced information.
OmniView itself shouldn't need to know anything about any given analysis-tool at compile-time, but should still be able to use an analysis-API. 
In order for OmniView to work this way, there shall be a structured architectural approach, to enable modular development.
This includes dataproducer devices as well as analysis-tools.

\section[Modules]{Modules of OmniView and their Role}
The Name OmniView actually only applies to the executable instance, that is in charge of displaying gathered data in a way similar to PicoScope or sigrok, whether it's live or archived data. 
Since (well-designed) modularity can help keeping the complexity of a system in check, the greater OmniView-Architecture is a project of the Bochumer AI-Group.
\\
A oscilloscope-like user-interface is the most generic way, to display data as a function over time. 
All measurement-values are a sample of a certain unit (sometimes even an SI-Unit) at a specific point in time, and thus projectable onto a plane with its unit in y- and the time in x-dimension. 
Displaying a multitude of different y-dimensions in an oscillogram is a well-proven design and has been implemented in several data-recorder software-suites. 
\\
Incoming live-data into the view is delivered via a websocket connection.
There won't be hard real-time guarantees in this interface. 
This websocket is provided by an entity that implements the concept of an epoch-server.
Soft real-time can be guaranteed to a certain extend in the internal structure of the epoch-server and its storage-interface.
Due to the nature of live-updates of a GUI, new updates are being received on a regular basis and prepended to the n-1 dataset.
Each update-object that contains data to be prepended, and displayed in the oscillogram is called a column. 
It derives its name from the property, that it holds several values of different data-channels in parallel.
The aggregation to a continuous stream of such pieces of information that belong to one channel is called a waveform. 
\\
An epoch-server has the ability to maintain multiple websocket connections, and an OmniView-Instance has the ability to maintain connections to multiple epoch-servers. 
Before a view can instantiate a connection with a websocket, it queries the epoch-servers REST-API, to receive a structure know as possibility-list.
This list contains all available devices, and all available transducers. 
A transducer is a function that works as a filter.
It implements a directed node, taking one or more waveforms as an input and having exactly one waveform as an output. 
Using transducers, a so-called (processing-)route can be constructed. 
A route is a directed graph composed from transducer-nodes. 
For further investigation see \ref{chap:WaveformProcessingNetwork}.
The structure inside the epoch server that defines which channels are being send out to a specific connection is called the connections visibility-list. 
This list in combination with a checkbox will also be displayed in the view.

\begin{figure}
    \includegraphics[\width=.9]{assets/overview.pdf}
    \caption{Overview of Modules}
\end{figure}

\section[Greater Picture]{OmniView and its Role in the Greater Picture}
Auto-Intern GmbH and their connected entities have been working on a unified architecture for measurement- and monitoring-devices since the early 2000s. 
Integrating measurement-systems into larger architectures is by no means a trivial task.
OmniView fits into the Grand-Unified-Monitoring-Architecture of Auto-Intern.


\begin{figure}
    \includegraphics[width=.9\textwidth]{./assets/pictures/overview.pdf}
    \caption[]{Brief overview of the proposed structure}
    \label{fig:overview}
\end{figure}


There shall be a unified way to interact with an abstract data-producer.
This includes devices such as:
\begin{enumerate}
    \item a USB-oscilloscope \gls{latex}
    \item a TCP/IP client, sending a continuous stream
    \item a USB-logic-analyzer
    \item a random-number-generator
    \item a filedescriptor
\end{enumerate}

Since it is not known at compile-time, which devices will be used at runtime, the code can't be linked statically into OmniView (or any other data-\gls{consumer} for that matter). 
Therefor an interface shall be defined, that gets used by the consumer, but the implementation of the data-handling ought to be provided in a dynamically linked library. 
From here on forward we will refer to this as \lstinline|DLL| even though \lstinline|.dll| and \lstinline|.so| are meant equally. 
If a windows \lstinline|.dll| or a linux \lstinline|.so| is meant specifically, please use the terms \lstinline|.dll| or \lstinline|.so|, otherwise \lstinline|DLL|. 
Be aware, that this \lstinline|DLL| does not necessarily constitue an aquivalent to an actual device-driver with a communication-channel to the systems kernel.
\\
It appears, that all dataproducers that are relevant for OmniView can be abstracted in a certain way, and thus share the same function-calls in a \lstinline|DLL|.
There are three requirements:
\begin{itemize}
    \item Grabbing the next part of the data-stream asynchronusly
    \item The data-producer providing meta-information about itself 
    \item Send control-data from the consumer to the data-producer
\end{itemize}
This interprocess-communication comes with some additional hurdles.

\section{Memory Management Ideology}
Not only does OmniView not know about which devices will be connected at runtime, it also does neither know about the amount of devices that will be attached, nor does it know what data-rate the producers will provide.
Due to this uncertainties, a strongly structured memory allocation ideology needs to be implemented in order to minimize error-prone sections in the applications code.


\section[UaDI]{Unified abstract Data\-producer Interface}
The \textit{Unified abstract Data\-producer Interface} is the protocol that specifies the interprocess\-communication between the con\-sumer and the device, using the \lstinline|DLL|. 

\begin{figure}
    \includegraphics[width=.9\textwidth]{./assets/pictures/interface.pdf}
    \caption[]{Coarse structure of the DLL-interface}
    \label{fig:dllinterface}
\end{figure}

\section{OmniView Userinterface}

The OmniView user-interface is written in C++, in a later process this will be replaced by JS. The Interface should contain 

\begin{itemize}
    \item an adjustable window where the data is displayed
    \item a general menu for settings, help and others 
    \item a menu for all extra windows 
    \item a window where the devices are displayed
    \item a window to upload the data
    \item a window to save the data 
    \item a window to check the old data analyses results 
    \item an exit button
\end{itemize}

and look like the following picture. 

% insert picture 
The detailed descriptions of the individual elements can be found below.

\subsection{Datawindow}

The following parts of the view should be adjustable: 

\begin{itemize}
    \item screensize
    \item scale of the x- and y-axis 
    \item fontsize 
\end{itemize}

The adjustments should be able to be made per mouse, mousepad, touchpad and with shortcuts. They also should be able to be set to a specific number. 
The user should be able to zoom in the dataset and cut out a specific part of the data. The image of the data should be able to be converted into a PDF. 

\subsection{Data upload window}

The upload window should be an extra window that does not overlap with the datawindow. It should contain an upload button and an exit button. The user should be able to put in the 
current dataset or another dataset from their database. 
The window should contain fields where the user can put in 
\begin{itemize}
    \item the analysis type
    \item the base data like the workplace of the measurement
    \item the reasons for the measurement
    \item if the measurement shows expected course or not 
\end{itemize}

\subsection{Saving data}

This section describes the saving process in the user interface and its key features. 
In \cite{fig:saveData} the visual representation of the menus can be found.

- The user should be able to press a save button after stopping a measurement 
- The save button should be visible before 
- After pressing the save button a window should open where the user can save their data 
- The window should include an input field for a storage path, the default should be a storage path to a “saves”-folder in the OmniScope-folder
- if more than one device is connected the user should be able to save more than one path by pressing a plus button at the bottom of the save window that saves the existing settings, closes the window and opens a copy of the save window
- next to the path there should be an drop down menu where the user can choose the devices that should be saved in the storage path via a checkmarkbox next to the devices  
- The user should be able to enter the vehicle type, the vin, a name for the measurement and the mileage of the car under the storage path field
- the use for the fields should stand as a string-placeholder in the input fields
- After entering the information the user should be able to press the save button at the end of the save window 
- The window should contain an exit button that closes the window when pressed 
- The data should be saved in a .csv file in the specified storage path
- The extra details the user entered should be placed in the header of the .csv file 

\begin{figure}
    \includegraphics[width=.9\textwidth]{./assets/pictures/SaveandOpenLucidChartSaveScreenshot.png}
    \caption[]{A visual representation of the saving process in the user interface}
    \label{fig:saveData}
\end{figure}

\subsection{Help menu}

The Helpmenu should contain 
\begin{itemize}
    \item a link to our ous
    \item a link to an Tutorial website
\end{itemize}




\chapter{The Waveform-Processing-Network}
\label{chap:WaveformProcessingNetwork}
A transducer is a directed node, and serves the function of forwarding and manipulating a value of a given waveform.
Routes are directed graphs, consisting of nodes of transducers and attributeless connections between them.
Routes can be dynamically constructed. 
The Waveform-Procesing-Network it is a global singleton static object, that gets default-constructed at the epoch-servers startup, and offers the routines .addRoute(Route\_t\&\& newRoute).
These transducer-nodes might take additional construction parameters. 
A route is to be read from right to left, since the interpretation can be ambigious reading from left to right. 
Routes can be configured via a REST-API POST method, that adds a specific route to a connection. 
\chapter[Further Devl]{Further Development of OmniView}
OmniView as it is right now has a limited shelf-life. 
The current version will be deprecated somewhere around summer of 2024.
It will be replaced by a more modular approach, that consists of two separate pieces, called OmniDaemon and OmniView 2.0.
OmniView 2.0 will be written in Angular.
OmniDaemon will still be written in C++ and use the same approach for data-acquisition. 
The interprocess-communication between the data and the view on the data will be implemented by using websockets for live-streaming, REST for settings and downloading whole files and MQTT-Publishing for sending out alarms and similar fire-and-forget messages. 


\printglossaries

\end{document}